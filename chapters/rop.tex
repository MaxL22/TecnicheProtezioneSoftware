% !TeX spellcheck = it_IT
\section{Return Oriented Programming ROP}
\label{sec:rop}

Prevenzione e attacchi si "inseguono" sempre:
\begin{itemize}
	\item \textbf{Difesa}: stack/heap non eseguibile per prevenire iniezione di codice. \textbf{Attacco}: jump/return to libc
    
	\item \textbf{Difesa}: nascondere gli indirizzi di memoria e return address con ASLR. \textbf{Attacco}: ricerca brute force (32 bit) o information leak (format string)
	
    \item \textbf{Difesa}: non usare codice in libc ma usare solo codice all'interno del text del programma, si riducono le funzioni di libreria legate all'utilizzo del processo stesso, vengono caricate solo le parti necessarie. \textbf{Attacco}: costruire le funzionalità che servono tramite Return Oriented Programming ROP
\end{itemize}

Il Return Oriented Programming ROP nasce dall'esigenza di limitare il numero di funzioni che un processo usa per la propria esecuzione. Introdotta per la prima volta nel 2007 da Hovav Shacham (\href{https://www.ush.it/team/ascii/geometry.pdf}{\textit{\texttt{The Geometry of Innocent Flesh on the Bone: Return-into-libc without Function \\ Calls}}}).

L'idea di base è: prendo pezzi di codice dalle funzioni già presenti in memoria e le unisco per "costruire" un attacco, ovvero il programma voluto dall'attaccante. 

Si può dimostrare che, con una codebase sufficiente, gli elementi costruibili sono Turing-compatibili. I pezzi di codice usati si chiamano \textbf{gadget}. Si tratta di una "nuova" tecnica per eseguire codice sfruttando una vulnerabilità (tipicamente sempre buffer overflow, necessita comunque di sovrascrivere un return address).

Le "sfide" di questo attacco sono 
\begin{itemize}
	\item trovare i gadget
    
	\item collegarli
\end{itemize}

Ma \textit{cosa sono i gadget}? Semplicemente sequenze di istruzioni (assembly) che terminano con un \texttt{ret}. 

Si "trasforma" il processo di esecuzione (il programma diventa una \href{https://en.wikipedia.org/wiki/Weird_machine}{\texttt{weird machine}}), lo stack diventa il "codice" per l'attaccante; non si può iniettare codice all'interno dello stack ma nella weird machine descritta:
\begin{itemize}
	\item \texttt{\%esp} diventa (una sorta di) program counter
    
	\item i gadget sono invocati tramite una \texttt{ret} (si parte dalla prima, quella sovrascritta da un buffer overflow per esempio)
	
    \item i gadget hanno parametri, passati tramite lo stack, quindi tramite \texttt{pop}, \dots 
\end{itemize}

Si ha una trasformazione della memoria del programma. L'idea è
\begin{itemize}
	\item prendere il codice che vogliamo eseguire
	
    \item emulare l'assembly tramite i gadget
\end{itemize}

I gadget trovati in memoria (pezzi di codice terminati da \texttt{ret}) vengono concatenati per simulare il comportamento voluto. L'overflow del buffer termina sovrascrivendo l'indirizzo del primo gadget in memoria. 

Esempio: volendo simulare il codice
\begin{itemize}[label*=]
	\item \texttt{mov \%edx, 5}
\end{itemize}

Avendo come gadget
\begin{itemize}[label*=, noitemsep]
	\item \texttt{pop \%edx}
	\item \texttt{ret}
\end{itemize}

E il valore 5 sullo stack, possiamo fare in modo che il valore puntato da \texttt{\%esp} sarà caricato in \texttt{\%edx} ed \texttt{\%esp} spostato sopra. 

Al termine di ogni gadget si ha il \texttt{ret}, fondamentale per concatenare i gadget. Cosa fa una \texttt{ret}? Prende il primo valore sullo stack (\texttt{pop} del valore puntato da \texttt{\%esp}) e prosegue da lì (codice puntato) l'esecuzione del programma.

Quindi, se sullo stack è presente l'indirizzo del gadget successivo, l'esecuzione proseguirà da lì.

In altre parole: cambio l'indirizzo di ritorno con l'indirizzo del primo gadget, sullo stack ci sarà la chain di return address dei gadget e i relativi parametri; ogni volta che ne viene eseguito uno (a partire dal primo), lo stack pointer scende e trova il successivo.

Esistono tool automatici per automatizzare la ricerca e unione dei gadget, chiamati ROP Compiler.

\begin{center}
	\begin{tabular}{| r c c l | r c c l |}
		\hline
		\multicolumn{4}{| c }{Sequenza di codice} & \multicolumn{4}{| c |}{Equivalente ROP} \\
		\hline
		\texttt{0x17f:} & \texttt{mov} & \texttt{\%eax,} & \texttt{[\%esp]} & \texttt{0x17f:} & \texttt{pop} & \texttt{\%eax} & \\
		& \texttt{mov} & \texttt{\%ebx,}  & \texttt{[\%esp+8]} & & \texttt{ret} & & \\
		& \texttt{mov} & \texttt{[\%ebx],} & \texttt{\%eax} & \texttt{\dots} &&& \\
		&&&& \texttt{0x20f:} & \texttt{pop} & \texttt{\%ebx} & \\
		&&&& & \texttt{ret} && \\
		&&&& \texttt{\dots} &&& \\
		&&&& \texttt{0x21a:} & \texttt{mov} & \texttt{[\%ebx],} & \texttt{\%eax} \\
		\hline
	\end{tabular}
\end{center}

Sostanzialmente, nello stack saranno presenti:
\begin{itemize}
	\item indirizzo del gadget
	\item parametro/i del gadget
	\item indirizzo del gadget
	\item \dots
\end{itemize} 
Quindi bisogna sovrascrivere lo stack con un layout di questo tipo.

\paragraph{Trovare i gadget:} I gadget per costruire un exploit possono essere trovati con una ricerca automatica del binario (cercando \texttt{ret} e andando a ritroso, \href{https://github.com/0vercl0k/rp}{\texttt{esempio di ROP gadget finder}}).

\textit{Ma quanti gadget sono presenti?} Ogni programma, in architettura Intel, può essere visto come $n$ rappresentazioni diverse: dato che è un'architettura \href{https://it.wikipedia.org/wiki/Complex_instruction_set_computer}{\texttt{CISC}} le operazioni possono avere diverse lunghezze: saltare in mezzo a un'istruzione porta a una codifica diversa del programma.

Esempio: se l'istruzione \texttt{a} ha opcode \texttt{0x0a0b} e l'istruzione \texttt{b} ha opcode più breve \texttt{0x0b}, "saltando" il primo byte dell'istruzione \texttt{a} ho effettivamente trovato un'istanza di istruzione \texttt{b}. In questo modo si possono trovare \texttt{ret} (o qualunque cosa) in maniera più semplice. Diventa più difficile su architetture RISC (tutti i byte di istruzione sono allineati).

I gadget sono sempre sufficienti per portare avanti un attacco? Generalmente sì, Shacham ha provato che per code base non triviali (e.g., libc), i gadget sono Turing completi.

Un ROP Compiler prende in input:
\begin{itemize}
	\item codice malevolo da eseguire ad alto livello
    
	\item programma vittima
\end{itemize}

E restituisce in output l'injection vector, ovvero il layout dello stack per effettuare l'attacco (eseguire il codice input). Prende il programma, definisce i gadget necessari, li trova nel programma e crea il layout dello stack.

\subsection{Blind ROP}

Si tratta di un attacco pubblicato da stanford (\href{http://www.scs.stanford.edu/brop/}{\texttt{qui la pagina}}) che mostra come il \textbf{ROP} si possa applicare \textbf{in condizioni reali}. Il contesto è: 
\begin{itemize}
	\item Remoto, si tratta di un server \href{https://nginx.org/}{\texttt{nginx}}
    
	\item L'attaccante non ha nè binario (programma eseguibile) nè source code
    
	\item ASLR, canary, DEP attivi
    
	\item Conoscenza di una vulnerabilità, da qualche parte (in questo caso vulnerabilità nota per la versione di nginx)
\end{itemize}

L'idea è: su un eseguibile PIE a 64 bit in esecuzione su un server, se in seguito a un crash il server riparte ma non ri-randomizza i valori si può: 
\begin{itemize}
	\item Leakare canary e return address dallo stack
	
    \item Trovare gadget (run-time) per leakare il binario
	
    \item Trovare i gadget per la shellcode
\end{itemize}

Lo \textbf{scopo} del programma è andare a \textbf{leakare il binario}, ovvero usare un gadget che va in memoria, prende il programma e lo restituisce all'attaccante. 

Per fare questo bisogna far fare al server una \texttt{write} su una socket \texttt{sd}, con buffer e relativa lunghezza, dove 
\begin{itemize}
	\item il buffer sarà l'indirizzo del programma, sconosciuto causa ASLR
	
    \item la lunghezza sarà la dimensione dell'applicazione, approssimativamente nota
\end{itemize}

Dato che a ogni connessione il server mantiene gli stessi valori (nuovo thread, stessi dati, fork del processo), anche se poi il thread crasha, le fasi sono: 
\begin{enumerate}
	\item Leakare la canary
    
	\item Defeat the ASLR
	
    \item Trovare in modo blind i gadget necessari per la \texttt{write}
	
    \item Ottenere il binario
\end{enumerate}

\paragraph{Leakare la canary:} Trovo la dimensione del buffer (tirando a indovinare), per poi provare a sovrascrivere un byte della canary alla volta, provando tutti i valori finché smette di crashare, per poi passare al byte successivo.

\paragraph{Defeat the ASLR:} Provo a indovinare il return address, in maniera simile a come fatto per la canary, bisogna scoprire \textit{più o meno} dove si trova il programma in memoria.

Una volta scoperta la canary, si può tirare a indovinare possibili return address finché non si trova uno degli indirizzi dov'è posizionato il programma (non crasha quando prova a uscire).

\paragraph{Blind search:} Bisogna trovare in modo blind i gadget che permettano di fare la \texttt{write}. Per una \texttt{write(sd, buffer, length)} le istruzioni che servono sono

\begin{tabular}{c l}
	\texttt{pop rdi} & \texttt{//sd} \\
	\texttt{pop rsi} & \texttt{//buffer} \\
	\texttt{pop rdx} & \texttt{//length} \\
	\texttt{pop rax} & \texttt{//write syscall num} \\
	\texttt{syscall} & \\
\end{tabular}

Quindi servono i gadget per fare

\begin{tabular}{c c}
	\texttt{pop rdi;} & \texttt{ret} \\
	\texttt{pop rsi;} & \texttt{ret} \\
	\texttt{pop rdx;} & \texttt{ret} \\
	\texttt{pop rax;} & \texttt{ret} \\
	\texttt{syscall} & \\
\end{tabular}

Il \texttt{sd} si può indovinare (8 bit, facile), ma questi gadget vanno trovati senza avere il binario. 

Per indovinare il gadget si può sfruttare la differenza di comportamento tra \texttt{pop} e altre istruzioni: \texttt{pop} sposta il registro \texttt{esp}.

Dopo aver trovato la canary si tira a indovinare, sovrascrivendo il return address, per trovare due gadget:
\begin{itemize}
	\item \textbf{idle gadget}: un indirizzo che mantiene aperta la socket (non crasha)
    
	\item \textbf{stop gadget}: un indirizzo che fa crashare la socket
\end{itemize}

I gadget reali si trovano sfruttando questi due: sullo stack metto, in ordine
\begin{enumerate}
	\item indirizzo del gadget cercato
	
    \item stop gadget
	
    \item idle gadget. 
\end{enumerate}

Provando a eseguire, il gadget cercato può essere:
\begin{itemize}
	\item un'istruzione normale (non \texttt{pop}): l'\texttt{esp} non viene spostato, il gadget eseguito dopo è lo stop e la socket crasha
    
	\item una \texttt{pop}: viene spostato l'\texttt{esp}, lo stop gadget non viene eseguito, la socket rimane aperta
\end{itemize}

In questo modo si possono mappare i gadget in memoria a run-time, si trovano le posizioni delle \texttt{pop}. 

Per capire su che registro viene fatta la \texttt{pop}: non lo capisco, si provano a caso le \texttt{pop} trovate e quando torna indietro qualcosa dalla socket ho beccato la combinazione giusta (ha fatto la \texttt{write}).

%TODO: maybe completa un po'
L'indirizzo per la \texttt{write} lo si trova tirando a indovinare nella GOT.

%End L7