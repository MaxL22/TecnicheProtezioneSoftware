% !TeX spellcheck = it_IT

\section{Memory Safety and Type Safety}

Tutti gli errori descritti nelle sezioni precedenti nascono da un \textbf{utilizzo errato della memoria}, nei linguaggi che lo permettono. Utilizzare \textbf{linguaggi memory safe} permetterebbe di evitare le problematiche descritte precedentemente, ma il degrado delle performance potrebbe non essere accettabile per alcuni casi d'uso.\\
Memory safety e type safety sono due proprietà intrinseche ad alcuni linguaggi di programmazione che permettono di bloccare la possibilità di effettuare gli attacchi descritti in precedenza.\\

Le \textbf{difese} possono essere a \textbf{livello} di: 
\begin{itemize}
	\item \textbf{compilatore}: come le canary, usate per rilevare una eventuale sovrascrittura del return address
	\item \textbf{sistema operativo}: come il Address Space Layout Randomization ASLR, che impedisce di sapere le posizioni esatte dei valori in memoria
	\item \textbf{architetturale}: componenti hardware che permettono di bloccare gli attacchi
\end{itemize}

Per contrastare ASLR sono nati gli attacchi Return Oriented Programming ROP, e per contrastare questi esiste la Control Flow Integrity CFI (controllo sulle transizioni del programma, se va "fuori dagli schemi" la transizione è bloccata, come ad esempio cambiando il return address). \\

\subsection{Memory Safety }
Sicurezza della memoria, si tratta di una proprietà fondamentale di alcuni linguaggi (detti memory safe). Un programma scritto in un linguaggio memory safe:
\begin{itemize}
	\item Permette di creare puntatori solo attraverso alcuni mezzi standard; si vogliono intercettare i punti di creazione dei puntatori
	\item Permette di usare puntatori solo per accedere a memoria che "appartiene" a quel puntatore; un puntatore che appartiene ad una zona di memoria deve accedere effettivamente a quella zona di memoria 
\end{itemize}
Combina le idee di temporal safety (accedere solo a memoria attualmente allocata/valida) e spatial safety (accedere solo a memoria valida).\\

\subsubsection{Spatial safety}
Ci permette di far valere la proprietà di sicurezza spaziale della memoria, ovvero andare a vedere che un puntatore non vada a puntare oltre una zona di memoria stabilita.\\

\paragraph{Fat Pointers:} Un puntatore non è più "solo un puntatore", ma diventa una tripla $(p,b,e)$ dove: 
\begin{itemize}
	\item $p$ è il puntatore effettivo
	\item $b$ è la base della zona di memoria a cui può accedere
	\item $e$ è l'estensione/limite della zona a cui può accedere
\end{itemize}

L'accesso è permesso se e solo se 
$$b \leq p \leq e - \text{\texttt{sizeof(typeof(} $p$ \text{\texttt{))}}} $$

Ogni puntatore ha un tipo, quindi "quanto si può spostare" è determinato anche dalla dimensione del tipo puntato (se è su int di 4 byte non posso puntare al penultimo byte, andrei fuori per gli ultimi 3). L'aritmetica sui puntatori modifica solo $p$, senza toccare $b$ ed $e$; $p$ si sposta, gli altri due rimangono lì a stabilire i limiti. Deve sempre valere la disuguaglianza.\\

I Fat Pointers sono puntatori che hanno dati aggiuntivi, come la tripla descritta in precedenza; metadati aggiunti ai puntatori.  Effettuare il controllo \textbf{ad ogni accesso} degrada le performance (soprattutto quando non è l'unico controllo da effettuare, questo è solo per la spatial safety). Il controllo è oneroso ed è necessaria una memoria aggiuntiva per i metadati legati ad ogni puntatore.\\
Il compilatore di un linguaggio memory safe deve instanziare memoria aggiuntiva per ogni puntatore e aggiungere il codice per i controlli, ad ogni accesso.\\

\newpage

\paragraph{Low Fat Pointers:} Si tratta di una variante che vuole ridurre l'overhead in termini di spazio e prestazioni. L'idea è quella di "codificare" all'interno dell'indirizzo stesso i metadati, la rappresentazione nativa del puntatore può inglobare i metadati. La disposizione della memoria viene usata per ricavare le informazioni prima salvate nei metadati in tempo costante.\\

La memoria virtuale viene divisa in "regioni", ciascuna riservata ad oggetti di dimensioni simili. Questo consente di dedurre la dimensione di un oggetto basandosi solo sull'indirizzo del puntatore (se appartiene a quella regione deve avere una certa dimensione).\\


\subsubsection{Temporal Safety}
Le regioni di memoria possono essere: 
\begin{itemize}
	\item \textbf{definite}: allocate e attive
	\item \textbf{non definite}: non inizializzate, non allocate o deallocate
\end{itemize}

Quando un puntatore punta ad una zona di memoria non definita è un problema. Per evitare errori serve tener traccia di dove un puntatore punta all'interno delle regioni di memoria. Dobbiamo evitare dangling pointers.\\
Esempio: 
\begin{minted}{c}
p = malloc(4)
s = p
free(p)
\end{minted}
\texttt{s} rimane dangling.\\

Serve una tabella di memoria per ogni puntatore che punta a tale zone. Quando la zona viene deallocata/diventa non definita, tutti i puntatori che fanno riferimento a quella zona devono essere resi non più validi. Nell'esempio precedente, \texttt{s} dovrebbe essere messo a \texttt{NULL} dopo la deallocazione. Questo richiede un controllo sulla validità del puntatore ad ogni dereferenziazione.\\

La combinazione di spatial e temporal safety si chiama \textbf{memory safety}. Il modo più semplice per ottenerla è utilizzare un linguaggio memory safe. C/C++ non sono memory safe, ma permettono di scrivere codice memory safe, il problema è che \textit{non ci sono garanzie}. Il compilatore potrebbe aggiungere codice per controllare le violazioni, ma rimane sempre il problema dell'inevitabile degrado delle performance (bisogna solo stabilire quanto e  se questo è accettabile).\\

% End L5
