% !TeX spellcheck = it_IT

\section{Introduzione}

\paragraph{Debugger:} Permette di analizzare un programma riga per riga, impostando break point ed analizzando la memoria durante l'esecuzione.

Alcuni comandi: 
\begin{itemize}
	\item \texttt{run r}: esegue il programma

	\item \texttt{step}: istruzione successiva (entrando nelle funzioni)

	\item \texttt{next}: istruzione successiva, ma le funzioni sono trattate come istruzione singola 

	\item \texttt{info registers}: per vedere stato dei registri

	\item \texttt{x/nfu}: per esaminare la memoria, dove \texttt{n} è il numero di unità da stampare, \texttt{f} il formato, \texttt{u} il tipo di unità

	\item \texttt{b addr}: imposta un breakpoint all'indirizzo \texttt{addr}, dove l'indirizzo può essere relativo, ad esempio \texttt{*foo+123} mette un breakpoint all'indirizzo \texttt{+123} della funzione \texttt{foo}

	\item \texttt{disassemble foo}: mostra le istruzioni della funzione \texttt{foo}

	\item \texttt{set what=val}: setta \texttt{what} al valore \texttt{val}
\end{itemize}

\paragraph{Ghidra:} Decompiler, prende l'output di un disassembler e cerca di ricostruire istruzioni di più alto livello in base a pattern noti. Dall'assembly cerca di ricostruire un codice (forse) leggibile.

\paragraph{Calling convention 64 bit:} Su architettura 64 bit, quando viene chiamata una funzione i valori vengono caricati all'interno dei registri, nell'ordine
\begin{center}
	\texttt{rdi, rsi, rdx, rcx, r8, r9}
\end{center}
Se ne sono presenti di più allora vengono caricati sullo stack.