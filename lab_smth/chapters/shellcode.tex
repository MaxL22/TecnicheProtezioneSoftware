% !TeX spellcheck = it_IT
\section{Shellcode Injection}

L'obiettivo è iniettare e far eseguire al programma una shellcode (o comunque codice arbitrario). Lo schema più semplice possibile è:
\begin{itemize}
	\item ho un buffer in cui posso fare overflow
	
    \item scrivo all'interno del buffer una shellcode/codice
	
    \item sovrascrivo il return address con l'indirizzo del buffer
\end{itemize}

Possibili problemi con l'esecuzione della shellcode: 
\begin{itemize}
	\item \textbf{stack non eseguibile}: protezione solitamente disabilitata per gli esercizi
	
    \item \textbf{shellcode più piccolo del buffer}: devo calcolare quanto padding serve per terminare il buffer ed arrivare al return address
	
    \item \textbf{non abbastanza spazio per lo shellcode}: più opzioni
	\begin{itemize}
		\item dividere lo shellcode in due parti, la prima avrà un'istruzione "finale" che fa saltare 16 byte in avanti per "skippare" \texttt{SFP} e \texttt{return\_address}, per poi continuare con il codice dello shellcode. Per saltare avanti
		\begin{itemize}
			\item \texttt{jmp 0x10} in avanti
		
        	\item \texttt{mv rax, rip; add rax, 0x10; mov rip, rax}
		\end{itemize}
        
		\item anziché dividere lo shellcode, si inizia con del padding fino a sovrascrivere l'indirizzo di ritorno, per poi scrivere lo shellcode completo al di sotto (o sopra, punti di vista; di solito sopra lo stack frame corrente c'è tutto lo spazio necessario); padding di \texttt{dim\_buffer+8} byte di \texttt{SFP + 8} byte \texttt{return\_address}
	\end{itemize}
\end{itemize}

Per costruire lo shellcode, msfvenom, a mano, exploitdb, ma per l'esame pwntools permette di crearne con shellcraft. Esempio pwntools: 
\begin{minted}{python}
context.arch = 'amd64'  # imposta l'architettura
shellcode = shellcraft.sh()
print(asm(shellcode))
\end{minted}

Esistono diversi comandi per generare diversi tipi di payload, \href{https://docs.pwntools.com/en/stable/shellcraft.html}{\texttt{visibili sulla documentazione}}.