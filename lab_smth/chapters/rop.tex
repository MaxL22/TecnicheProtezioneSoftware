% !TeX spellcheck = it_IT
\section{Return Oriented Programming ROP}

Voglio creare una catena di gadget per eseguire codice arbitrario. La sfida è capire cosa fare con i gadget a disposizione. In generale, nelle challenge, la catena va creata a partire da: 
\begin{center}
	\texttt{payload = b'A'*buf\_offset + gadget1 + param1 + gadget2 + param2 + \dots}
\end{center}
Fino ad avere tutti i gadget e parametri sullo stack.\\

\paragraph{Trovare i gadget:} Esistono tool ropper, ropgadgets, \dots (cerca). Altrimenti si può fare a mano (you psycho).\\
Esempio 
\begin{center}
	\texttt{ropper -f file}
\end{center}
Così li sputa tutti, per specificare quale gadget: \texttt{--search "pop rdi"}.\\

Pwntools invece ha un modulo ROP
\begin{center}
	\texttt{rop = ROP(binario)}
\end{center}
e permette di trovare i gadget con (esempio come sopra):
\begin{center}
	\texttt{rop.rdi.address}
\end{center}
Potrebbe trovarne più di uno, in tal caso è da specificare quale usare.\\

Pwntools permette anche di creare catene di rop: dopo aver caricato il modulo rop sopra, posso chiamare una funzione con dei parametri specifici:
\begin{center}
	\texttt{rop.win\_stage\_4(4)}
\end{center}
per passare il parametro \texttt{4} alla funzione \texttt{win\_stage\_4()}.\\

Per vedere la catena creata
\begin{center}
	\texttt{rop.dump()}
\end{center}
Per mandarla 
\begin{center}
	\texttt{rop.chain()}
\end{center}

Se non ho una win da chiamare:
\begin{itemize}
	\item inietto una shellcode sullo stack: ma potrebbe essere non eseguibile
	\item uso le syscall: potrei avere gadget per fare chiamate di sistema, ce ne sono per tutti i gusti
\end{itemize}