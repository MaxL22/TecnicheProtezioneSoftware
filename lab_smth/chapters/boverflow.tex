% !TeX spellcheck = it_IT

\section{Buffer Overflow}

Per gli esercizi di pwncollege, in generale si ha a disposizione un buffer e bisogna far eseguire una funzione \texttt{win()}, in qualche modo (dipende dall'esercizio, ma generalmente sovrascrivendo l'indirizzo di ritorno della funzione \texttt{challenge()} con l'indirizzo di \texttt{win()}), al fine di ottenere una flag.

Per capire la dimensione del buffer: 
\begin{itemize}
	\item Ghidra: la notazione di ghidra fa riferimento all'inizio dello stack, se il buffer è chiamato \texttt{local\_a8}, l'inizio dello stack (e di conseguenza il punto subito dopo l'indirizzo di ritorno della funzione) si trova \texttt{0xa8} bit prima del buffer (i.e., devo sovrascrivere 176 byte, 168 di buffer e cose varie poi 8 di indirizzo di ritorno)
    
	\item gdb: guardo l'indirizzo del buffer, cercando dove viene popolato rispetto al RBP (poi ci sono 8 byte di Saved Frame Pointer SFP, poi il return address). Non so come si faccia tbh, dovrei scoprirlo.
	
    \item Stringhe cicliche: scrivo $n$ indirizzi di memoria diversi (valori casuali) e guardo il core dump: lo stato dei registri mostra il valore dell'instruction pointer, ovvero quale dei $n$ diversi indirizzi ha fatto crashare il programma, se è il $10^o$ indirizzo allora la distanza è $10 \cdot 8$, potremmo farlo a mano, oppure usare pwntools
\end{itemize}

\begin{minted}{python}
# Per definire l'eseguibile
elf = ELF(path) 
# Con suid attivo non fornisce core dump
io = elf.process(setuid=False)
# Genera stringa ciclica, 512 byte in blocchi da 8 
io.sendline(cyclic(512, 8)) 
# Aspetta il crash
io.wait() 
# Estrapola l'offset da inizio stack, 
#    in base all'indirizzo da cui è crashato il programma,
#    con indirizzi da 8 byte
print(cyclic_find(io.corefile.fault_addr, n=8))
\end{minted}

Quest'ultimo  meccanismo non funziona con canary o stack protection. Checksec sul binario (da pwntools) restituisce le protezioni sull'eseguibile. 

Se all'interno della funzione \texttt{win()} c'è il check di un parametro all'inizio della funzione, posso:
\begin{itemize}
	\item usare dei "gadget" (to be continued)
	
    \item andare ad un istruzione dopo il controllo, al posto dell'indirizzo iniziale della funzione \texttt{win()} scrivo l'indirizzo di una istruzione all'interno della funzione, ma dopo il check
\end{itemize}

\subsection{Randomizzazione degli indirizzi ASLR}

Generalmente attivo di default, l'ASLR \textbf{randomizza la posizione del programma in memoria}, di conseguenza rendendo casuali return address e indirizzo del buffer. 

Ci possono essere diverse implementazioni, con le relative difficoltà, ma in generale è implementato a livello \textbf{kernel}: ogni volta che un programma viene lanciato gli viene assegnata una pagina diversa; gli indirizzi vengono calcolati a runtime tramite offset.

Rimane noto l'offset all'interno del programma, ovvero le ultime 3 cifre in esadecimale dell'indirizzo (12 bit, le pagine sono allineate a \texttt{0x1000}). Si può sfruttare questa cosa sovrascrivendo solo gli ultimi 2 byte dell'indirizzo di ritorno e presupponendo che la funzione da eseguire non sia "troppo lontana" dal resto. In questo modo sarà casuale solo l'ultima cifra in esadecimale dovuta al paging, basta ripetere l'esecuzione "qualche volta" (probabilità $1/16$) per indovinare anche l'ultimo carattere.

\subsection{Canary}

Se presente una canary, dobbiamo leakarla per poi sovrascriverla. Generalmente gli esercizi hanno un \texttt{\%s} e una "backdoor" che fa ripartire la challenge. 

Facendo overflow del buffer fino alla canary (+1 byte per sovrascrivere il primo \texttt{0x00} della canary, sempre presente) e non inserendo un terminatore di stringa il \texttt{\%s} stamperà il valore della canary, inseribile all'esecuzione successiva della challenge.